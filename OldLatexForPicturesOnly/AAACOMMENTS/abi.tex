\documentclass[12pt,a4paper]{article}
\usepackage[latin1]{inputenc}
\usepackage{calc}
\usepackage{setspace}
\usepackage{graphicx}
\usepackage{multicol}
\usepackage[normalem]{ulem}
%% Please set your language here
\usepackage[english]{babel}
\usepackage{color}
\usepackage{hyperref}



\begin{document}

[DAK: Hello, Dr. Ducasse.  I will insert each comment after the sentence or phrase it refers to.  I will enclose each comment within square brackets and my initials.  Iwill explain each kind of markup where I first use it.  I{f the system is unsatisfactory, p}lease describe its shortcomings, or suggest how to improve it. /DAK]





About this book 





Knowledge is only one part of understanding. Genuine understanding comes from hands on experience.  S. Papert 








Goal 





The goal of this book is to explain elementary programming concepts\textbf{ [}such as loops, abstractions, composition, and conditionals\textbf{] }to novices of all ages. 





[DAK:  Before I comment on a phrase{ in your text, I will mark it }by {bolding it, and either underlining it or }enclosing it in square brackets.  {I will mark suggested deletions by bracketing, bolding, underlining and striking out.  Generally I will insert further comment for a suggestion only if I think some explanation is necessary. }





The introduction of technical terms in the first sentence of your introduction seems unnecessary and possibly intimidating.  I suggest





The goal of this book is to explain elementary programming concepts to novices of all ages. 


\begin{flushleft}

\end{flushleft}


\begin{flushleft}
{(I will highlight my suggested rewording by bolding the font.)  Or  to suggest the relative unimportance of the sub-clause, make it an aside by parenthesizing it:}
\end{flushleft}


\begin{flushleft}

\end{flushleft}


\begin{flushleft}
The goal of this book is to explain elementary programming concepts (such as loops, abstractions, composition, and conditionals) to novices of all ages. 
\end{flushleft}


\begin{flushleft}

\end{flushleft}


\begin{flushleft}
{The original sentence is linguistically and logically correct in your original, so with this comment I am acting more like an 'editor' than a 'proofreader'.  }
\end{flushleft}


\begin{flushleft}
{However I shouldn't presume to offer editorial comment until you invite it, or before I understand what tone and style you want.  From here on I will restrict my role to proofreader for language and readability issues (grammar, confusing or unclear phrasing, vocabulary, usage, meaning, connotation, idiom etc.) until you ask otherwise. /DAK]}
\end{flushleft}


\begin{flushleft}

\end{flushleft}


I also believe that learning by experimenting and solving problems is \textbf{[}\textbf{the}\textbf{]}


central to human knowledge acquisition. 





Therefore I propose to \textbf{[understand] present }the concepts \textbf{[based on] through} simple problems. 





[DAK: 'understand' applies to the student, not the teacher.  I suggest 'explain', 'present' or 'introduce'.  'Based on' is a static relationship that isn't appropriate to the active concept in this sentence.  I suggest 'through' or 'using'.  So the sentence might read





\textbf{Therefore I propose to present the concepts through simple problems.} /DAK]





My ultimate goal is to teach you object-oriented programming because it provides an excellent metaphor for teaching programming. However, teaching object-oriented programming requires some basic notions of abstraction\textbf{[}\textbf{s}\textbf{]}. 





Therefore, I wrote this book to present these basic programming concepts with the special perspective that this book is the first of a series of two books. The current book is completely self-contained and does not require\textbf{ [] you }to read {the next one. }





[DAK: I mark suggested insertions in-line and bolded, immediately following empty brackets. /DAK]








The second book introduces a new robot that can be programmed. It focuses on\textbf{ [intermediary] intermediate-level }topics such as\textbf{ [finding path in maze] finding a path through a maze }{or drawing fractals. }





It also acts as a companion book for\textbf{ [the persons] people }who want {to know more, who want to adapt the environment to their own needs. It introduces object-oriented programming. }


\begin{flushleft}

\end{flushleft}


[DAK: 'the persons' is permissible but unidiomatic.  'people who want', or 'those who want' are more common.  Other phrases are also possible.  The meanings are nearly the same; the tone varies slightly, mostly in formality. /DAK]





\begin{flushleft}

\end{flushleft}


Audience 





The ideal reader I have in mind is a person that wants to have fun programming. This person may be a teenager or an adult, a teacher in high-school, or somebody willing to teach programming to kids in any organization. This person does not have to be fluent in programming in any language. 





The material of this book has been primarily developed for my\textbf{ [wife who] wife, who }is a \textbf{[physic] physics }and mathematics teacher in a\textbf{ [french] French }school (\textbf{[] where the }students are between 11\textbf{ [to] and }15 years old). 





\textbf{[Late] In late }1998, my wife had to teach computer\textbf{ [sciences] science }and we got frustrated by the lack of\textbf{ [adapted] appropriate }material. 


We were dreaming about a way to teach a process of facing problems and finding solutions. My wife started to teach HTML, Word and similar \textbf{[not exciting] }\textbf{{unexciting }}topics\textbf{ [], }and she was \textbf{\uline{absolutely not satisfied}}. 





{[DAK: 'not exciting' and '}{absolutely not satisfied' are }{slangy.  The first  doesn't read smoothly in this sentence, so I suggested a change. } The second reads OK, and seems to fit your desired tone, so I didn't suggest a change.  If you ask me to eliminate slang I will try, but it would be difficult to remove all slang and still be informal and idiomatic.  See my comment and question at the end of this doc. /DAK]


\begin{flushleft}

\end{flushleft}


She was surprised by the fact that a lot of material was simply ad-hoc and \textbf{[not promoting] }\textbf{{didn't promote }}any scientific attitude.





In addition we were aware of the work on Logo and liked the idea of \textbf{[}\textbf{the}\textbf{]}


experimentation as a basis for learning. We were aware that Smalltalk was influenced by the ideas of Papert and Logo\textbf{[], }and \textbf{[}by the fact\textbf{]} that it originated from\textbf{ [a research project to teach] }research on teaching programming to kids. Moreover Smalltalk has a simple syntax that mimics natural language. [] \{paragraph break\}


[DAK: I will mark suggested insertions of non-textual items like breaks by inserting the name in curly braces. /DAK]





At that time Squeak arrived\textbf{ [in] at }a mature state and books started to\textbf{ [be] become }available in late 1999. So I started and wrote the present book. 





The environments I use in this book and its companion book\textbf{ [is] are }fully working\textbf{[], }and went\textbf{ [over] through }several iterations\textbf{ [and] of }improvements based on the feedback we got from teachers.  A\textbf{ [constant point] guiding rule }in our work has been to modify\textbf{ [as few as possible the Squeak environment as] the Squeak environment as little as possible, as }our goal is that readers \textbf{[can extend and develop other ideas] will be able to extend these ideas and develop others}.





\begin{flushleft}
in Squeak.  \textbf{[Still has] This has had }for example the consequence that we did not provide an adapted debugger that\textbf{ [would] could }have hidden some irrelevant details. 
\end{flushleft}


\begin{flushleft}

\end{flushleft}





[DAK; Please send me your feedback about these comments, and your instructions and preferences for how I proofread later pages.  I am interested in your suggestions about formatting, which issues you are interested in and which I shouldn't waste my time on, etc.  /DAK]


\begin{flushleft}

\end{flushleft}

\end{document}
{\rtf1\ansi\ansicpg1252\deff0 {\fonttbl {\f0\fnil\fcharset0\fprq0\fttruetype Times New Roman;} {\f1\fnil\fcharset0\fprq0\fttruetype NULL;} {\f2\fnil\fcharset0\fprq0\fttruetype Dingbats;} {\f3\fnil\fcharset0\fprq0\fttruetype Symbol;} {\f4\fnil\fcharset0\fprq0\fttruetype Chapter;} {\f5\fnil\fcharset0\fprq0\fttruetype Bitstream Vera Sans;} {\f6\fnil\fcharset0\fprq0\fttruetype Courier New;} {\f7\fnil\fcharset0\fprq0\fttruetype Section;}} {\colortbl \red0\green0\blue0; \red255\green255\blue255;} {\stylesheet {\s19\sl240\slmult1\fi-431\li720\sbasedon20 Lower Roman List;} {\s21\sl240\slmult1\tx431\sbasedon13\snext20 Numbered Heading 1;} {\s22\sl240\slmult1\tx431\sbasedon14\snext20 Numbered Heading 2;} {\s28\sl240\slmult1\fi-431\li720 Square List;} {\*\cs9\sl240\slmult1\sbasedon20 Endnote Text;} {\s23\sl240\slmult1\tx431\sbasedon15\snext20 Numbered Heading 3;} {\s24\sl240\slmult1\fi-431\li720 Numbered List;} {\*\cs8\sl240\slmult1\fs20\super Endnote Reference;} {\s4\sl240\slmult1\tx1584\sbasedon21\snext20 Chapter Heading;} {\s7\sl240\slmult1\fi-431\li720 Diamond List;} {\s5\sl240\slmult1\fi-431\li720 Dashed List;} {\s34\sl240\slmult1\fi-431\li720\sbasedon24 Upper Roman List;} {\s16\sl240\slmult1\fi-431\li720 Heart List;} {\*\cs30\sl240\slmult1\f0\fs28\b\lang1033\sbasedon20 Suggestion;} {\s1\sl240\slmult1\li1440\ri1440\sa117\sbasedon20 Block Text;} {\*\cs26\sl240\slmult1\f0\fs28\b\ul\lang1033\sbasedon20 Referent;} {\s33\sl240\slmult1\fi-431\li720\sbasedon24 Upper Case List;} {\s3\sl240\slmult1\fi-431\li720 Bullet List;} {\s12\sl240\slmult1\fi-431\li720 Hand List;} {\*\cs11\sl240\slmult1\fs20\sbasedon20 Footnote Text;} {\s13\sl240\slmult1\sb440\sa60\f5\fs34\b\sbasedon20\snext20 Heading 1;} {\s14\sl240\slmult1\sb440\sa60\f5\fs28\b\sbasedon20\snext20 Heading 2;} {\s15\sl240\slmult1\sb440\sa60\f5\fs24\b\sbasedon20\snext20 Heading 3;} {\s31\sl240\slmult1\fi-431\li720 Tick List;} {\s20\sl240\slmult1\f0\fs24\lang1033 Normal;} {\s18\sl240\slmult1\fi-431\li720\sbasedon24 Lower Case List;} {\*\cs6\sl240\slmult1\f0\fs28\b\ul\strike\lang1033\sbasedon20 Deletion;} {\*\cs10\sl240\slmult1\fs20\super Footnote Reference;} {\s27\sl240\slmult1\tx1584\sbasedon21\snext20 Section Heading;} {\s17\sl240\slmult1\fi-431\li720 Implies List;} {\s2\sl240\slmult1\fi-431\li720 Box List;} {\s29\sl240\slmult1\fi-431\li720 Star List;} {\s25\sl240\slmult1\f6\sbasedon20 Plain Text;} {\s32\sl240\slmult1\fi-431\li720 Triangle List;}} \kerning0\cf0\ftnbj\fet2\ftnstart1\ftnnar\aftnnar\ftnstart1\ftnrestart\ftnrstpg\aftnstart1\aftnrestart\aendnotes\aenddoc{\info}\deftab720\viewkind1\paperw12240\paperh15840\margl1440\margr1440\widowctl \sectd\sbknone\colsx360\pgncont\ltrsect \pard\plain\ltrpar\ql\s20\sl240\slmult1\itap0{\f0\fs24\lang1033{\*\listtag0}\ltrch [DAK: Hello, Dr. Ducasse.  I will insert each comment after the sentence or phrase it refers to.  I will enclose each comment within square brackets and my initials.  Iwill explain each kind of markup where I first use it.  I}{\f0\fs24\lang1033{\*\listtag0}f the system is unsatisfactory, p}{\f0\fs24\lang1033{\*\listtag0}lease describe its shortcomings, or suggest how to improve it. /DAK]} \par\pard\plain\ltrpar\ql\s20\sl240\slmult1\itap0{\f0\fs24\lang1033{\*\listtag0} \par}\pard\plain\ltrpar\ql\s20\sl240\slmult1\itap0{\f0\fs24\lang1033{\*\listtag0}\ltrch About this book } \par\pard\plain\ltrpar\ql\s20\sl240\slmult1\itap0{\f0\fs24\lang1033{\*\listtag0} \par}\pard\plain\ltrpar\ql\s20\sl240\slmult1\itap0{\f0\fs24\lang1033{\*\listtag0}\ltrch Knowledge is only one part of understanding. Genuine understanding comes from hands on experience.  S. Papert } \par\pard\plain\ltrpar\ql\s20\sl240\slmult1\itap0{\f0\fs24\lang1033{\*\listtag0} \par}\pard\plain\ltrpar\ql\s20\sl240\slmult1\itap0{\f0\fs24\lang1033{\*\listtag0} \par}\pard\plain\ltrpar\ql\s20\sl240\slmult1\itap0{\f0\fs24\lang1033{\*\listtag0}\ltrch Goal } \par\pard\plain\ltrpar\ql\s20\sl240\slmult1\itap0{\f0\fs24\lang1033{\*\listtag0} \par}\pard\plain\ltrpar\ql\s20\sl240\slmult1\itap0{\f0\fs24\lang1033{\*\listtag0}\ltrch The goal of this book is to explain elementary programming concepts}{\f0\fs24\b\lang1033{\*\listtag0} [}{\f0\fs28\b\ul\strike\lang1033{\*\listtag0}such as loops, abstractions, composition, and conditionals}{\f0\fs24\b\lang1033{\*\listtag0}] }{\f0\fs24\lang1033{\*\listtag0}to novices of all ages. } \par\pard\plain\ltrpar\ql\s20\sl240\slmult1\itap0{\f0\fs24\lang1033{\*\listtag0} \par}\pard\plain\ltrpar\ql\s20\sl240\slmult1\itap0{\f0\fs24\lang1033{\*\listtag0}\ltrch [DAK:  Before I comment on a phrase}{\f0\fs24\lang1033{\*\listtag0} in your text, I will mark it }{\f0\fs24\lang1033{\*\listtag0}by }{\f0\fs24\lang1033{\*\listtag0}bolding it, and either underlining it or }{\f0\fs24\lang1033{\*\listtag0}enclosing it in square brackets.  }{\f0\fs24\lang1033{\*\listtag0}I will mark suggested deletions by bracketing, bolding, underlining and striking out.  Generally I will insert further comment for a suggestion only if I think some explanation is necessary. } \par\pard\plain\ltrpar\ql\s20\sl240\slmult1\itap0{\f0\fs24\lang1033{\*\listtag0} \par}\pard\plain\ltrpar\ql\s20\sl240\slmult1\itap0{\f0\fs24\lang1033{\*\listtag0}\ltrch The introduction of technical terms in the first sentence of your introduction seems unnecessary and possibly intimidating.  I suggest} \par\pard\plain\ltrpar\ql\s20\sl240\slmult1\itap0{\f0\fs24\lang1033{\*\listtag0} \par}\pard\plain\ltrpar\ql\s20\sl240\slmult1\itap0{\f0\fs24\b\lang1033{\*\listtag0}\ltrch The goal of this book is to explain elementary programming concepts to novices of all ages. } \par\pard\plain\ltrpar\ql\s20\sl240\slmult1\itap0{\f0\fs24\lang1033{\*\listtag0} \par}\pard\plain\ltrpar\ql\s20\sl240\slmult1\itap0{\f0\fs24\lang1033{\*\listtag0}\ltrch (I will highlight my suggested rewording by bolding the font.)  Or  to suggest the relative unimportance of the sub-clause, make it an aside by parenthesizing it:} \par\pard\plain\ltrpar\ql\s20\sl240\slmult1\itap0{\f0\fs24\lang1033{\*\listtag0} \par}\pard\plain\ltrpar\ql\s20\sl240\slmult1\itap0{\f0\fs24\b\lang1033{\*\listtag0}\ltrch The goal of this book is to explain elementary programming concepts (such as loops, abstractions, composition, and conditionals) to novices of all ages. } \par\pard\plain\ltrpar\ql\s20\sl240\slmult1\itap0{\f0\fs24\lang1033{\*\listtag0} \par}\pard\plain\ltrpar\ql\s20\sl240\slmult1\itap0{\f0\fs24\lang1033{\*\listtag0}\ltrch The original sentence is linguistically and logically correct in your original, so with this comment I am acting more like an 'editor' than a 'proofreader'.  } \par\pard\plain\ltrpar\ql\s20\sl240\slmult1\itap0{\f0\fs24\lang1033{\*\listtag0}\ltrch However I shouldn't presume to offer editorial comment until you invite it, or before I understand what tone and style you want.  From here on I will restrict my role to proofreader for language and readability issues (grammar, confusing or unclear phrasing, vocabulary, usage, meaning, connotation, idiom etc.) until you ask otherwise. /DAK]} \par\pard\plain\ltrpar\ql\s20\sl240\slmult1\itap0{\f0\fs24\lang1033{\*\listtag0} \par}\pard\plain\ltrpar\ql\s20\sl240\slmult1\itap0{\f0\fs24\lang1033{\*\listtag0}\ltrch I also believe that learning by experimenting and solving problems is }{\f0\fs24\b\lang1033{\*\listtag0}[}{\f0\fs28\b\ul\strike\lang1033{\*\listtag0}the}{\f0\fs24\b\lang1033{\*\listtag0}]} \par\pard\plain\ltrpar\ql\s20\sl240\slmult1\itap0{\f0\fs24\lang1033{\*\listtag0}\ltrch central to human knowledge acquisition. } \par\pard\plain\ltrpar\ql\s20\sl240\slmult1\itap0{\f0\fs24\lang1033{\*\listtag0} \par}\pard\plain\ltrpar\ql\s20\sl240\slmult1\itap0{\f0\fs24\lang1033{\*\listtag0}\ltrch Therefore I propose to }{\f0\fs24\b\lang1033{\*\listtag0}[understand] present }{\f0\fs24\lang1033{\*\listtag0}the concepts }{\f0\fs24\b\lang1033{\*\listtag0}[based on] through}{\f0\fs24\lang1033{\*\listtag0} simple problems. } \par\pard\plain\ltrpar\ql\s20\sl240\slmult1\itap0{\f0\fs28\b\ul\lang1033{\*\listtag0} \par}\pard\plain\ltrpar\ql\s20\sl240\slmult1\itap0{\f0\fs24\lang1033{\*\listtag0}\ltrch [DAK: 'understand' applies to the student, not the teacher.  I suggest 'explain', 'present' or 'introduce'.  'Based on' is a static relationship that isn't appropriate to the active concept in this sentence.  I suggest 'through' or 'using'.  So the sentence might read} \par\pard\plain\ltrpar\ql\s20\sl240\slmult1\itap0{\f0\fs24\lang1033{\*\listtag0} \par}\pard\plain\ltrpar\ql\s20\sl240\slmult1\itap0{\f0\fs24\b\lang1033{\*\listtag0}\ltrch Therefore I propose to present the concepts through simple problems.}{\f0\fs24\lang1033{\*\listtag0} /DAK]} \par\pard\plain\ltrpar\ql\s20\sl240\slmult1\itap0{\f0\fs24\lang1033{\*\listtag0} \par}\pard\plain\ltrpar\ql\s20\sl240\slmult1\itap0{\f0\fs24\lang1033{\*\listtag0}\ltrch My ultimate goal is to teach you object-oriented programming because it provides an excellent metaphor for teaching programming. However, teaching object-oriented programming requires some basic notions of abstraction}{\f0\fs24\b\lang1033{\*\listtag0}[}{\f0\fs28\b\ul\strike\lang1033{\*\listtag0}s}{\f0\fs24\b\lang1033{\*\listtag0}]}{\f0\fs24\lang1033{\*\listtag0}. } \par\pard\plain\ltrpar\ql\s20\sl240\slmult1\itap0{\f0\fs24\lang1033{\*\listtag0} \par}\pard\plain\ltrpar\ql\s20\sl240\slmult1\itap0{\f0\fs24\lang1033{\*\listtag0}\ltrch Therefore, I wrote this book to present these basic programming concepts with the special perspective that this book is the first of a series of two books. The current book is completely self-contained and does not require}{\f0\fs24\b\lang1033{\*\listtag0} [] you }{\f0\fs24\lang1033{\*\listtag0}to read }{\f0\fs24\lang1033{\*\listtag0}the next one. } \par\pard\plain\ltrpar\ql\s20\sl240\slmult1\itap0{\f0\fs24\lang1033{\*\listtag0} \par}\pard\plain\ltrpar\ql\s20\sl240\slmult1\itap0{\f0\fs24\lang1033{\*\listtag0}\ltrch [DAK: I mark suggested insertions in-line and bolded, immediately following empty brackets. /DAK]} \par\pard\plain\ltrpar\ql\s20\sl240\slmult1\itap0{\f0\fs24\lang1033{\*\listtag0} \par}\pard\plain\ltrpar\ql\s20\sl240\slmult1\itap0{\f0\fs24\lang1033{\*\listtag0} \par}\pard\plain\ltrpar\ql\s20\sl240\slmult1\itap0{\f0\fs24\lang1033{\*\listtag0}\ltrch The second book introduces a new robot that can be programmed. It focuses on}{\f0\fs24\b\lang1033{\*\listtag0} [intermediary] intermediate-level }{\f0\fs24\lang1033{\*\listtag0}topics such as}{\f0\fs24\b\lang1033{\*\listtag0} [finding path in maze] finding a path through a maze }{\f0\fs24\lang1033{\*\listtag0}or drawing fractals. } \par\pard\plain\ltrpar\ql\s20\sl240\slmult1\itap0{\f0\fs24\lang1033{\*\listtag0} \par}\pard\plain\ltrpar\ql\s20\sl240\slmult1\itap0{\f0\fs24\lang1033{\*\listtag0}\ltrch It also acts as a companion book for}{\f0\fs24\b\lang1033{\*\listtag0} [the persons] people }{\f0\fs24\lang1033{\*\listtag0}who want }{\f0\fs24\lang1033{\*\listtag0}to know more, who want to adapt the environment to their own needs. It introduces object-oriented programming. } \par\pard\plain\ltrpar\ql\s20\sl240\slmult1\itap0{\f0\fs24\lang1033{\*\listtag0} \par}\pard\plain\ltrpar\ql\s20\sl240\slmult1\itap0{\f0\fs24\lang1033{\*\listtag0}\ltrch [DAK: 'the persons' is permissible but unidiomatic.  'people who want', or 'those who want' are more common.  Other phrases are also possible.  The meanings are nearly the same; the tone varies slightly, mostly in formality. /DAK]} \par\pard\plain\ltrpar\ql\s20\sl240\slmult1\itap0{\f0\fs24\lang1033{\*\listtag0} \par}\pard\plain\ltrpar\ql\s20\sl240\slmult1\itap0{\f0\fs24\lang1033{\*\listtag0} \par}\pard\plain\ltrpar\ql\s20\sl240\slmult1\itap0{\f0\fs24\lang1033{\*\listtag0}\ltrch Audience } \par\pard\plain\ltrpar\ql\s20\sl240\slmult1\itap0{\f0\fs24\lang1033{\*\listtag0} \par}\pard\plain\ltrpar\ql\s20\sl240\slmult1\itap0{\f0\fs24\lang1033{\*\listtag0}\ltrch The ideal reader I have in mind is a person that wants to have fun programming. This person may be a teenager or an adult, a teacher in high-school, or somebody willing to teach programming to kids in any organization. This person does not have to be fluent in programming in any language. } \par\pard\plain\ltrpar\ql\s20\sl240\slmult1\itap0{\f0\fs24\lang1033{\*\listtag0} \par}\pard\plain\ltrpar\ql\s20\sl240\slmult1\itap0{\f0\fs24\lang1033{\*\listtag0}\ltrch The material of this book has been primarily developed for my}{\f0\fs24\b\lang1033{\*\listtag0} [wife who] wife, who }{\f0\fs24\lang1033{\*\listtag0}is a }{\f0\fs24\b\lang1033{\*\listtag0}[physic] physics }{\f0\fs24\lang1033{\*\listtag0}and mathematics teacher in a}{\f0\fs24\b\lang1033{\*\listtag0} [french] French }{\f0\fs24\lang1033{\*\listtag0}school (}{\f0\fs24\b\lang1033{\*\listtag0}[] where the }{\f0\fs24\lang1033{\*\listtag0}students are between 11}{\f0\fs24\b\lang1033{\*\listtag0} [to] and }{\f0\fs24\lang1033{\*\listtag0}15 years old). } \par\pard\plain\ltrpar\ql\s20\sl240\slmult1\itap0{\f0\fs24\lang1033{\*\listtag0} \par}\pard\plain\ltrpar\ql\s20\sl240\slmult1\itap0{\f0\fs24\b\lang1033{\*\listtag0}\ltrch [Late] In late }{\f0\fs24\lang1033{\*\listtag0}1998, my wife had to teach computer}{\f0\fs24\b\lang1033{\*\listtag0} [sciences] science }{\f0\fs24\lang1033{\*\listtag0}and we got frustrated by the lack of}{\f0\fs24\b\lang1033{\*\listtag0} [adapted] appropriate }{\f0\fs24\lang1033{\*\listtag0}material. } \par\pard\plain\ltrpar\ql\s20\sl240\slmult1\itap0{\f0\fs24\lang1033{\*\listtag0}\ltrch We were dreaming about a way to teach a process of facing problems and finding solutions. My wife started to teach HTML, Word and similar }{\f0\fs24\b\lang1033{\*\listtag0}[not exciting] }{\f0\fs24\b\lang1033{\*\listtag0}unexciting }{\f0\fs24\lang1033{\*\listtag0}topics}{\f0\fs24\b\lang1033{\*\listtag0} [], }{\f0\fs24\lang1033{\*\listtag0}and she was }{\f0\fs24\b\ul\lang1033{\*\listtag0}absolutely not satisfied}{\f0\fs24\lang1033{\*\listtag0}. } \par\pard\plain\ltrpar\ql\s20\sl240\slmult1\itap0{\f0\fs24\lang1033{\*\listtag0} \par}\pard\plain\ltrpar\ql\s20\sl240\slmult1\itap0{\f0\fs24\lang1033{\*\listtag0}\ltrch [DAK: 'not exciting' and '}{\f0\fs24\lang1033{\*\listtag0}absolutely not satisfied' are }{\f0\fs24\lang1033{\*\listtag0}slangy.  The first  doesn't read smoothly in this sentence, so I suggested a change. }{\f0\fs24\lang1033{\*\listtag0} The second reads OK, and seems to fit your desired tone, so I didn't suggest a change.  If you ask me to eliminate slang I will try, but it would be difficult to remove all slang and still be informal and idiomatic.  See my comment and question at the end of this doc. /DAK]} \par\pard\plain\ltrpar\ql\s20\sl240\slmult1\itap0{\f0\fs24\lang1033{\*\listtag0} \par}\pard\plain\ltrpar\ql\s20\sl240\slmult1\itap0{\f0\fs24\lang1033{\*\listtag0}\ltrch She was surprised by the fact that a lot of material was simply ad-hoc and }{\f0\fs24\b\lang1033{\*\listtag0}[not promoting] }{\f0\fs24\b\lang1033{\*\listtag0}didn't promote }{\f0\fs24\lang1033{\*\listtag0}any scientific attitude.} \par\pard\plain\ltrpar\ql\s20\sl240\slmult1\itap0{\f0\fs24\lang1033{\*\listtag0} \par}\pard\plain\ltrpar\ql\s20\sl240\slmult1\itap0{\f0\fs24\lang1033{\*\listtag0}\ltrch In addition we were aware of the work on Logo and liked the idea of }{\f0\fs24\b\lang1033{\*\listtag0}[}{\f0\fs28\b\ul\strike\lang1033{\*\listtag0}the}{\f0\fs24\b\lang1033{\*\listtag0}]} \par\pard\plain\ltrpar\ql\s20\sl240\slmult1\itap0{\f0\fs24\lang1033{\*\listtag0}\ltrch experimentation as a basis for learning. We were aware that Smalltalk was influenced by the ideas of Papert and Logo}{\f0\fs24\b\lang1033{\*\listtag0}[], }{\f0\fs24\lang1033{\*\listtag0}and }{\f0\fs24\b\lang1033{\*\listtag0}[}{\f0\fs28\b\ul\strike\lang1033{\*\listtag0}by the fact}{\f0\fs24\b\lang1033{\*\listtag0}]}{\f0\fs24\lang1033{\*\listtag0} that it originated from}{\f0\fs24\b\lang1033{\*\listtag0} [a research project to teach] }{\f0\fs24\lang1033{\*\listtag0}research on teaching programming to kids. Moreover Smalltalk has a simple syntax that mimics natural language. [] \{paragraph break\}} \par\pard\plain\ltrpar\ql\s20\sl240\slmult1\itap0{\f0\fs24\lang1033{\*\listtag0}\ltrch [DAK: I will mark suggested insertions of non-textual items like breaks by inserting the name in curly braces. /DAK]} \par\pard\plain\ltrpar\ql\s20\sl240\slmult1\itap0{\f0\fs24\lang1033{\*\listtag0} \par}\pard\plain\ltrpar\ql\s20\sl240\slmult1\itap0{\f0\fs24\lang1033{\*\listtag0}\ltrch At that time Squeak arrived}{\f0\fs24\b\lang1033{\*\listtag0} [in] at }{\f0\fs24\lang1033{\*\listtag0}a mature state and books started to}{\f0\fs24\b\lang1033{\*\listtag0} [be] become }{\f0\fs24\lang1033{\*\listtag0}available in late 1999. So I started and wrote the present book. } \par\pard\plain\ltrpar\ql\s20\sl240\slmult1\itap0{\f0\fs24\lang1033{\*\listtag0} \par}\pard\plain\ltrpar\ql\s20\sl240\slmult1\itap0{\f0\fs24\lang1033{\*\listtag0}\ltrch The environments I use in this book and its companion book}{\f0\fs24\b\lang1033{\*\listtag0} [is] are }{\f0\fs24\lang1033{\*\listtag0}fully working}{\f0\fs24\b\lang1033{\*\listtag0}[], }{\f0\fs24\lang1033{\*\listtag0}and went}{\f0\fs24\b\lang1033{\*\listtag0} [over] through }{\f0\fs24\lang1033{\*\listtag0}several iterations}{\f0\fs24\b\lang1033{\*\listtag0} [and] of }{\f0\fs24\lang1033{\*\listtag0}improvements based on the feedback we got from teachers.  A}{\f0\fs24\b\lang1033{\*\listtag0} [constant point] guiding rule }{\f0\fs24\lang1033{\*\listtag0}in our work has been to modify}{\f0\fs24\b\lang1033{\*\listtag0} [as few as possible the Squeak environment as] the Squeak environment as little as possible, as }{\f0\fs24\lang1033{\*\listtag0}our goal is that readers }{\f0\fs24\b\lang1033{\*\listtag0}[can extend and develop other ideas] will be able to extend these ideas and develop others}{\  f0\fs24\lang1033{\*\listtag0}.} \par\pard\plain\ltrpar\ql\s20\sl240\slmult1\itap0{\f0\fs24\lang1033{\*\listtag0} \par}\pard\plain\ltrpar\ql\s20\sl240\slmult1\itap0{\f0\fs24\lang1033{\*\listtag0}\ltrch in Squeak.  }{\f0\fs24\b\lang1033{\*\listtag0}[Still has] This has had }{\f0\fs24\lang1033{\*\listtag0}for example the consequence that we did not provide an adapted debugger that}{\f0\fs24\b\lang1033{\*\listtag0} [would] could }{\f0\fs24\lang1033{\*\listtag0}have hidden some irrelevant details. } \par\pard\plain\ltrpar\ql\s20\sl240\slmult1\itap0{\f0\fs24\lang1033{\*\listtag0} \par}\pard\plain\ltrpar\ql\s20\sl240\slmult1\itap0{\f0\fs24\lang1033{\*\listtag0} \par}\pard\plain\ltrpar\ql\s20\sl240\slmult1\itap0{\f0\fs24\lang1033{\*\listtag0}\ltrch [DAK; Please send me your feedback about these comments, and your instructions and preferences for how I proofread later pages.  I am interested in your suggestions about formatting, which issues you are interested in and which I shouldn't waste my time on, etc.  /DAK]} \par\pard\plain\ltrpar\ql\s20\sl240\slmult1\itap0{\f0\fs24\lang1033{\*\listtag0}}} 