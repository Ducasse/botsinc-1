Hello Stef,

I don't have the TeX source for these chapters:
String.tex, Installation.tex,
InstallationTroubleShootings.tex.  Please send me a
copy if you want my suggestions.

==================================================

In previous chapters I have hesitated to offer
suggestions for "heavy" revision for several reasons:

1. I can't mimic your style and voice very well, so
the contrast would distract readers.  It would be best
for you to rewrite my rewrites --- but that puts more
work on you.

2. The issues in heavy revision are quite debatable,
and the discussions they are likely to provoke could
be inconclusive.  We might spend time without reaching
an agreement or resulting in any improvement.

3. I greatly respect your classroom experience
(directly and through teachers you know personally)
and your feel for students reactions to the material,
specific orders of presentation, and wordings.  I do
not have that experience, so my rewrites will be blind
to many merits of your original, and to many defects
in my suggestions.  My high-level suggestions are
riskier, and will require more cautious consideration
from you, than my (relatively safe) low-level edits
for language and vocabulary issues.

However since time is running out, if you don't object
I will start including more rewritten paragraphs.  I
will mark them with @@dank: ...@@ and insert them as
alternate versions, adjacent  to my low-level edits
(marked as before).  

Here's a sample from the first paragraph of Chapter
18, "Conditions".

Original version:
-----------------
Up until now the programs we defined were executing
\textit{all} the messages that composed them one after
the other. We had no way to define that certain
messages could only be executed when certain
conditions were met. In this chapter and the following
one we introduce an important programming concept: the
notion of \emph{conditional} \index{condition}
execution, \ie\ the fact a certain piece of code is
executed when a given condition holds.

Suggested "light" revision, addressing language and
wording issues only (this is the level of editing I
have been doing):
------------------------------------------------------
Up until now the programs we defined \replace{were
executing}{executed} \textit{all} the messages
\replace{that composed them}{they contained,} one
after the other. We had no way to
\replace{define}{say} that certain messages
\replace{could}{should} only be executed \replace{when
certain}{under some} conditions\replace{ were met}{,
but not in others}. \replace{In this}{This} chapter
and the \replace{following one we}{next} introduce an
important programming concept: the notion of
\emph{conditional} \index{condition} execution,
\replace{\ie\}{i.e.} the fact \add{that} a certain
piece of code \replace{is executed}{executes only}
when a \replace{given}{specified} condition
\replace{holds}{is true}. 

The result of the "light" revision (so you don't have
to run tex to read it):
-----------------------------------------------------
Up until now the programs we defined executed
\textit{all} the messages they contained, one after
the other. We had no way to say that certain messages
should only be executed under some conditions, but not
in others. This chapter and the next introduce an
important programming concept: the notion of
\emph{conditional} \index{condition} execution, i.e.
the fact that a certain piece of code executes only
when a specified condition is true. 

"Heavy" revision, addressing pedagogical issues,
content, order etc.:
-----------------------------------------------------
@@dank: I think this paragraph could be simpler,
clearer, livelier, and fit its emphases more closely
to the essential details.  Here's a possible revision:
"All of the programs we have seen so far executed
\textit{all} the code they contained \textit{every
time} they ran. They did not make any decisions about
whether or not to execute an expression, or send a
message. This chapter shows a way a program can make a
choice: it can decide to execute an expression one
time, and not execute it another time, depending on
the situation when the program runs.  Programmers call
this general idea \emph{conditional execution}
\index{condition}, because some code executes only
when a specified condition occurs."@@

======================================================

I will summarize the issues and possible solutions
abstractly in the introduction, but describing them
completely would be a daunting, lengthy job; and I
would still need to supplement any abstract
description with a specific proposal to make my
meaning clear.  One danger then is that my rewrite
will seem too assertive and prescriptive.  I hope we
both remember that even if I identify a problem or
issue that we agree on --- there are always
alternative solutions that are about equally good.  I
hope to provoke you to your own improvements.

More later
- Daniel K.
 